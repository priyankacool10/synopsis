\chapter{System Design}

\section{Design Approach}
A design approach is a general philosophy that may or may not include a guide for specific methods. Some are to guide the overall goal of the design. Other approaches are to guide the tendencies of the designer. A combination of approaches may be used if they don't conflict.\\
\emph{Function Oriented Design Approach:}\\
Function Oriented Design Approach is partitioning of a design into subsystems and modules, with each one handling one or more functions. Contrast with object-oriented design, data-structure-oriented design. \\

This application project uses function oriented design approach. Every module and sub modules are made, based on their functionality.
These modules are designed and implemented separately and then they are integrated together to form the desired application.

\section{Detail Design}
The detailed design of this application is as follow:
\begin{enumerate}
\item \textbf{\emph{Registering a User:}}\\
The first step in this application is to get the users registered to both GCM Server and to Remote Web Server. For this, user will provide all the necessary details and press the register button. The request will first go to Google Cloud Messaging Server. GCM Server will provide
the registration id for that device. After that, all the information along with registration id is stored on Web Server and the user gets registered.

\item \textbf{\emph{User Login:}}\\
After registering, the user is allowed to log in. Username and password after validating at client side, is sent to server side to authentication. After authentication response is sent by the server to client, and then user gets logged in.

\item \textbf{\emph{Viewing the Notices:}}\\
At the first time, when you are using this application for the first time, it will fetch all the notices from server. In all the other case, all previous notices are fetched from application's own database stored inside client mobile. It then checks for new notices from the server. 
If there are new notices on the server, it will fetch all those notices.

\item \textbf{\emph{Searching a Notice:}}\\
The user is able to search the notice in listview depending on the title of the notice. It helps user to get the desired notice instantly.

\item \textbf{\emph{Deleting a Notice:}}\\
If the user does not want some notices, he/she can delete it from their phone. There will be no effect on server entry.

\item \textbf{\emph{Posting a Notice:}}\\
If a user is an admin, he is able to post the notice. In order to post the notices, he has three option. One option is that, he can post a simple text notice. Another option allows him to send some attachment image with the notice. In this, he has two options.
Either he can pick the image from the gallery or he can click a picture on the spot by using camera.
After that, press the post button to post the notice.

\item \textbf{\emph{Notification Buzz:}}\\
As soon as the admin post a notice, the script will run with which request is made by GCM Server to WebServer for all the registered IDs.
After getting all the registered IDs, notification is sent to all the users registered with this application.
Notification has a tune and vibration that runs whenever there is a notification received by the user from GCM Server.

\item \textbf{\emph{Reset Password:}}\\
This application also has the facility to reset the password. If one user has forgot his passwor, he/she can rest the password by giving
his username or email address. The user will be given a page in which he can set his new password. Forgotten password will be updated with the new one on the server.

\end{enumerate}

\begin{figure}[H]
\centering \includegraphics[scale=0.9]{image/detaildesign.png}
\caption{Detailed Design}
\end{figure}
\pagebreak

\section{System Design}
The system design can be clearly explained from the following diagrams:\\
\textbf{\emph{Use Case Diagram:}}\\

A Use Case diagram at its simplest is a representation of a user's interaction with the system and depicting the specifications of a use case. A use case diagram can portray the different types of users of a system and the various ways that they interact with the system. This type of diagram is typically used in conjunction with the textual use case and will often be accompanied by other types of diagrams as well.
\\
There are two types of user in this application, user and admin. Following depits their use case diagram:

\begin{figure}[H]
\centering \includegraphics[scale=0.5]{image/usecase1.png}
\caption{Use Case Diagram For User}
\end{figure}

This diagram is showing what a normal user can do with this application. The user can login, after that he can view the notices, delete the notices and can search for particular notices.

\begin{figure}[H]
\centering \includegraphics[scale=0.5]{image/usecase2.png}
\caption{Use Case Diagram For Admin}
\end{figure}

This diagram shows the priveleges of admin. An admin can post the notice in addition to viewing and deleting it. He can also post images as an attachment to the notices. Images can be choosen from the gallery or he can click the picture instantly usinng this application.






%%%%%%%%%%%%%%%%%%%%%%%%%%%%%%%%%%%%%%%%%%%%%%%%%%%%%%%%%%%%%%%%%%%%%%%%%%%%%%%%%%%%%%%%%%%%%%%%%%%%%%%%%%%%%%%%%%%%%%%%%%%%%%%%%
\pagebreak
\section{User Interface Design}
User Interface Design means the design of application with  which the user interacts. So it should be kept in mind that UI should be very simple and easy to use. It should be simple enough in look and feel also.
\begin{figure}[H]
\centering \includegraphics[scale=0.5]{image/ui1.png}
\caption{Landing Page}
\end{figure}

This page is the first page which is presented to the user. It has two buttons, to login or to register.

\begin{figure}[H]
\centering \includegraphics[scale=0.5]{image/ui2.png}
\caption{Registration Page}
\end{figure}

This is the registration page where the user get himself registered with the servers of this application.

\begin{figure}[H]
\centering \includegraphics[scale=0.5]{image/ui3.png}
\caption{Login Page}
\end{figure}

This is the login page where user enters his username and password in order to access the notices.

\begin{figure}[H]
\centering \includegraphics[scale=0.5]{image/ui4.png}
\caption{Dashboard Of Notices}
\end{figure}

This is the main dashboard, which receives all the notices from the user. The user can click on any notice, to see its details.
In addition to it, user can search and delete the notices.

\begin{figure}[H]
\centering \includegraphics[scale=0.5]{image/ui5.png}
\caption{Post Notice Page}
\end{figure}

This page is only for the admin. Only admin can post the notices. He can choose images as attachment to notices.

\begin{figure}[H]
\centering \includegraphics[scale=0.5]{image/ui6.png}
\caption{Reset Password Page}
\end{figure}

This is Reset Password page where the user can reset the password in order if he forgots his password.

%%%%%%%%%%%%%%%%%%%%%%%%%%%%%%%%%%%%%%%%%%%%%%%%%%%%%%%%%%%%%%%%%%%%%%%%%%%%%%%%%%%%%%%%%%%%%%%%%%%%%%%%%%%%%%%%%%%%%%%%%%%%%%
\pagebreak
\section{Database Design}
Database design is the process of producing a detailed data model of a database. This logical data model contains all the needed logical and physical design choices and physical storage parameters needed to generate a design in a data definition language, which can then be used to create a database. A fully attributed data model contains detailed attributes for each entity.\\

The term database design can be used to describe many different parts of the design of an overall database system. Principally, and most correctly, it can be thought of as the logical design of the base data structures used to store the data. In the relational model these are the tables and views. In an object database the entities and relationships map directly to object classes and named relationships. However, the term database design could also be used to apply to the overall process of designing, not just the base data structures, but also the forms and queries used as part of the overall database application within the database management system (DBMS).\\

The process of doing database design generally consists of a number of steps which will be carried out by the database designer. Usually, the designer must:
\begin{itemize}
\item Determine the relationships between the different data elements.
\item Superimpose a logical structure upon the data on the basis of these relationships.
\end{itemize}



%%%%%%%%%%%%%%%%%%%%%%%%%%%%%%%%%%%%%%%%%%%%%%%%%%%%%%%%%%%%%%%%%%%%%%%%%%%%%%%%%%%%%%%%%%%%%%%%%%%%%%%%%%%%%%%%%%%%%%%%%%%%%
\pagebreak
\section{Methodology}
The methodology used in project is Agile Software Development. Agile Software Development methodology is used especially for software development, that is characterized by the division of tasks into short phases of work and frequent reassessment and adaptation of plans.\\

It is for a project that needs extreme agility in requirements. The key features of agile are its short-termed delivery cycles (sprints), agile requirements, dynamic team culture, less restrictive project control and emphasis on real-time communication.\\

Agile software development is a group of software development methods based on iterative and incremental development, in which requirements and solutions evolve through collaboration between self-organizing, cross-functional teams. It promotes adaptive planning, evolutionary development and delivery, a time-boxed iterative approach, and encourages rapid and flexible response to change. It is a conceptual framework that promotes foreseen tight iterations throughout the development cycle.\\
\begin{figure}[H]
\centering \includegraphics[scale=0.9]{image/agile.png}
\caption{Agile Software Development Methodology}
\end{figure}


The Manifesto of Agile Software Development context are:
\begin{itemize}
\item \emph{Individuals and interactions –} In agile development, self-organization and motivation are important, as are interactions like co-location and pair programming.
\item \emph{Working software –} Working software will be more useful and welcome than just presenting documents to clients in meetings.
\item \emph{Customer collaboration –} Requirements cannot be fully collected at the beginning of the software development cycle, therefore continuous customer or stakeholder involvement is very important.
\item \emph{Responding to change –} Agile development is focused on quick responses to change and continuous development.
\end{itemize}
